\section{Introdução}
A disciplina Sinais e Sistemas visa apresentar conceitos e formulação matemática para o entendimento de sinais; sejam eles unidimensionais ou multidimensionais.

A teoria de análise de sistemas, especialmente os lineares, é aplicada no contexto da engenharia para proporcionar uma base teórica e prática para operações realizadas por sistemas. Isso inclui a manipulação de sinais digitais utilizando ferramentas computacionais e a análise de sistemas tanto em tempo contínuo quanto discreto para extrair parâmetros relevantes.

O enfoque é fixar conceitos e metodologias que possibilitem estudos avançados em áreas como Princípios de Controle e Telecomunicações, e Processamento Digital de Sinais, incluindo processamento de imagens e compressão e codificação de sinais e vídeos.

Assim, essa prova visa a consolidação do conteúdo aprendido de forma que processamento de sinais de ECG são realizados e representações gráficas são feitas para a comprovação da teoria posta em prática.